\documentclass[11pt]{article}
% default ORG mode 
\usepackage[utf8]{inputenc}  \usepackage[T1]{fontenc} \usepackage{fixltx2e}
\usepackage{graphicx}        \usepackage{longtable}   \usepackage{float}
\usepackage{wrapfig}         \usepackage{soul}        \usepackage{textcomp}
\usepackage{marvosym}        \usepackage{wasysym}     \usepackage{latexsym}
\usepackage{amsmath}         \usepackage{amssymb}     \usepackage{color}
\usepackage{rotating}        \usepackage{hyperref}    \usepackage{xcolor} 
\usepackage{palatino}
% my customizations
\usepackage{enumitem}
\usepackage{amsthm}
\tolerance=1000
%\providecommand{\alert}[1]{\textbf{#1}}
\usepackage[top=1in, bottom=1in, left=1in, right=1in]{geometry}

\title{Dodson \& Poston Exercise VII.1.4}
\author{Peter Mao, $\ldots$}
\date{\today}

\begin{document}

\maketitle
\pagestyle{empty}

\begin{abstract}
  In-progress solution.  Feel free to add/comment/disparage.
\end{abstract}

Let $A$ be a linear map $T \to L(T;T')$ and define
\begin{align*}
  A' \colon T \times T &\to T'\\
                 (x,y) &\mapsto (A(x))y.
\end{align*}
Prove:

\begin{itemize}
\item[\textbf{(a)}] $A'$ is bilinear
\item[\emph{Solution}] Show that $A'(x+x',y+y') = A'(x,y) + A'(x,y') + A'(x',y) +A'(x',y')$, and
  scalar multiples of the arguments will follow.

\item[\textbf{(b)}] The map $\Phi \colon L(T;L(T;T')) \to L^2(T;T') \colon A \mapsto A'$ is an isomorphism.
\item[\emph{Solution}] Let $\dim T = n$ and $\dim T' = m$.  Then $A$ is a linear map from $n$ to $m
  \times n$ dimensions, while $A'$ is a bilinear map from $n \times n$ dimensions to $m$ dimensions.
  Therefore $\dim A = \dim A' = n^2m$, which implies that $\Phi$ is surjective.  Now, if we can show
  that $\ker \Phi = 0$ (which would imply injectivity )then $\Phi$ is an isomorphism.

  Lemma/Definition: if $k$-linear map $Ax = 0 \forall x \in X$, then $A$ is the zero map.
  
  The kernel of $\Phi$ is 0 if $A' = 0 \implies A = 0$.  If $A' = 0$ then $A'(x,y) = 0 \forall x,y
  \in T$.  Using the definition of $A'$, we also have $(A(x))y = 0 \forall x,y \implies A(x) = 0
  \forall x$; therefore, $A = 0$.  This shows that $\ker \Phi = 0$, so $\Phi$ is an isomorphism.

\item[\textbf{(c)}] (extend to higher orders)
\item[\emph{Solution}] Proof is exaclty the same with $(x_2,\ldots,x_k)$ standing in for $y$.
  
\end{itemize}
\end{document}



%% commonly used blocks
\begin{quote}
\end{quote}

\begin{proof}
\end{proof}

\begin{align}
\end{align}

\begin{equation}
\end{equation}

\begin{remark}
\end{remark}

