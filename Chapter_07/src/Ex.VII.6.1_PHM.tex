\documentclass[11pt]{article}
% default ORG mode 
\usepackage[utf8]{inputenc}  \usepackage[T1]{fontenc} \usepackage{fixltx2e}
\usepackage{graphicx}        \usepackage{longtable}   \usepackage{float}
\usepackage{wrapfig}         \usepackage{soul}        \usepackage{textcomp}
\usepackage{marvosym}        \usepackage{wasysym}     \usepackage{latexsym}
\usepackage{amsmath}         \usepackage{amssymb}     \usepackage{color}
\usepackage{rotating}        \usepackage{hyperref}    \usepackage{xcolor} 
\usepackage{palatino}
% my customizations
\usepackage{enumitem}
\usepackage{amsthm}
\tolerance=1000
%\providecommand{\alert}[1]{\textbf{#1}}
\usepackage[top=1in, bottom=1in, left=1in, right=1in]{geometry}

\title{Commentary on Dodson \& Poston Exercise VII.6.1}
\author{Peter Mao, $\ldots$}
\date{\today}

\begin{document}

\maketitle
\pagestyle{empty}

\begin{abstract}
  Demonstrate the connection between curves and vector fields in the examples of
  section VII.6.01.
\end{abstract}

\section{1a: $c(t) = \frac{t}{1 - t^2}$}

The velocity, as a function of $t$ is simple to calculate, as it is just the
derivative $\frac{dc}{dt} = \frac{1 + t^2}{(1 - t^2)^2}$.  The problem is to
write the velocity as a function of position, rather than time.  In this case,
we need to invert $c(t)$ into $t(c)$.

Suppressing the function notation on $c$, we can rewrite the expression for the
curve as
\begin{align}
  ct^2 + t - c = 0.
\end{align}
By the quadratic formula, we have $t$ as a function of $c$:
\begin{align}
  t = \frac{-1 + \sqrt{1 + 4c^2}}{2c}.
\end{align}
Only the positive square root is valid, as the negative branch gives times
outside of the domain $]-1,1[$.

    From here, we just plug in the above results:
    \begin{align}
      v(c(t)) = c^*(t)/\vec{e_1} &=  \frac{1 + t^2}{(1 - t^2)^2} \\
      &=  \frac{t^2}{(1 - t^2)^2} +  \frac{1}{(1 - t^2)^2} \\
      &=  c^2 + \frac{c^2}{t^2} \\
      &=  c^2 + \frac{4c^4}{(-1 + \sqrt{1 + 4c^2})^2} \\
      &=  c^2 + \frac{4c^4}{1 + 1 + 4c^2 - 2\sqrt{1 + 4c^2}} \\
    v(c)  &=  c^2 + \frac{2c^4}{1 + 2c^2 - \sqrt{1 + 4c^2}} 
    \end{align}
    as desired.
    
\section{1b, TBD.}

\end{document}



%% commonly used blocks
\begin{quote}
\end{quote}

\begin{proof}
\end{proof}

\begin{align}
\end{align}

\begin{equation}
\end{equation}

\begin{remark}
\end{remark}

