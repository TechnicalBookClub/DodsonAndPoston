\documentclass[11pt]{article}
% default ORG mode 
\usepackage[utf8]{inputenc}  \usepackage[T1]{fontenc} \usepackage{fixltx2e}
\usepackage{graphicx}        \usepackage{longtable}   \usepackage{float}
\usepackage{wrapfig}         \usepackage{soul}        \usepackage{textcomp}
\usepackage{marvosym}        \usepackage{wasysym}     \usepackage{latexsym}
\usepackage{amsmath}         \usepackage{amssymb}     \usepackage{color}
\usepackage{rotating}        \usepackage{hyperref}    \usepackage{xcolor} 
\usepackage{palatino}
% my customizations
\usepackage{enumitem}
\usepackage{amsthm}
\tolerance=1000
%\providecommand{\alert}[1]{\textbf{#1}}
\usepackage[top=1in, bottom=1in, left=1in, right=1in]{geometry}

\title{Dodson \& Poston Exercise VII.1.1}
\author{Peter Mao, Barney Tam, $\ldots$}
\date{\today}

\begin{document}

\maketitle
\pagestyle{empty}

\begin{abstract}
  In-progress solution.  Feel free to add/comment/disparage.
\end{abstract}

\begin{itemize}
\item
  [\textbf{(a)}] [Given] Hausdorff topological spaces $X,Y$ and any
  map (not necessarily continuous of everywhere defined) $f: X \to Y$,
  define

  $lim_{x\to p}(f(x)) = q$ if and only if for any neighborhood $N(q)$
  we can find a neighborhood $N(p)$ such that, if $x \in N(p)$ and
  $f(x)$ is defined, then $f(x) \in N(q)$.

\item [\emph{Comment}]
  What is there to prove or show here?  Are there any counterexamples
  that show how this can go wrong?

\item
  [\textbf{(b)}] If $X$ is the set of natural numbers $1,2,3,\ldots$
  together with one extra element which we label $\infty$, find a
  topology on $X$ which makes Definition VI.2.01 (pg 125-126, sequence
  of points and the limit of a sequence) a special case of the one
  above (a).

\item
  [\emph{Solution}]
Without loss of generality, we shall assume a metric topology for $X$.
In VI.2.01, a sequence is defined as a mapping $S: X \to Y$,
so we want to find a metric (distance function) as defined in VI.1.02
(pg 116) that makes the sequence definition a special case of (a).

Define the metric
\begin{alignat}{1}
  d \colon  X \times X &\to \mathbb{R} \\
            (m,n)      &\mapsto \bigg|\frac{1}{m} - \frac{1}{n}\bigg|
\end{alignat}
with $\frac{1}{\infty} = 0$. This satifies the conditions for a metric:
\begin{enumerate}[label=\roman*), itemsep=0pt]
\item $d(m,n) = d(n,m)$
\item $d(m,n) = 0 \iff m = n$
\item $d(m,n) \le d(m,r) + d(r,n)$
\end{enumerate}

The third condition requires a quick check.  If $m < r < n$, we have
equality because the sign of the $r$ term will differ in the two terms
on the right side when the absolute-value is ``removed.''  If $m < n <
r$ (the case $r < m < n$ proceeds similarly), then we use the fact
that $d(m,r) = d(m,n) + d(n,r)$, so that the triangle inequality now
reads
\begin{align}
  d(m,n) & \le d(m,r) + d(r,n) = d(m,n) + d(n,r) + d(r,n) \\
         & \le d(m,n) + 2d(n,r)
\end{align}
which holds since $d(n,r) > 0$.

Since $d$ maps into $\mathbb{R}$, we can use the usual topology on
$\mathbb{R}$.  Relating this back to part (a), $p = \infty$, $f = S$,
and $q$ is the limit of $S$.
\end{itemize}


\end{document}



%% commonly used blocks
\begin{quote}
\end{quote}

\begin{proof}
\end{proof}

\begin{align}
\end{align}

\begin{equation}
\end{equation}
