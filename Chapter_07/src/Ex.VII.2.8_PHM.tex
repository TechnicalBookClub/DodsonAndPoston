\documentclass[11pt]{article}
% default ORG mode 
\usepackage[utf8]{inputenc}  \usepackage[T1]{fontenc} 
\usepackage{graphicx}        \usepackage{longtable}   \usepackage{float}
\usepackage{wrapfig}         \usepackage{soul}        \usepackage{textcomp}
\usepackage{marvosym}        \usepackage{wasysym}     \usepackage{latexsym}
\usepackage{amsmath}         \usepackage{amssymb}     \usepackage{color}
\usepackage{rotating}        \usepackage{hyperref}    \usepackage{xcolor} 
\usepackage{palatino}
% my customizations
\usepackage{enumitem}
\usepackage{amsthm}
\tolerance=1000
%\providecommand{\alert}[1]{\textbf{#1}}
\usepackage[top=1in, bottom=1in, left=1in, right=1in]{geometry}

\title{Commentary on Dodson \& Poston Exercise VII.2.8}
\author{Peter Mao, $\ldots$}
\date{\today}

\begin{document}

\maketitle
\pagestyle{empty}

\begin{abstract}
  Not a complete solution to this problem, just comments and sketches of concrete
  examples to illustrate the problem.  Feel free to add/comment/disparage.
\end{abstract}

\section{Commentary on Ex. VII.2.8}
D\&P use $x$ both as the point in the manifold $M$ and as the coordinate labels
in the affine spaces that the charts in $M$ and $M'$ map to.  This leads to some
confusion in reading the problem, so in this document, I take $p \in M$, $N$ is
a neighborhood of $p$ and leave the $x^i \in X, X'$ as coordinate in the affine
spaces.

\begin{quote}
  Suppose that $f \colon M \rightarrow M'$ is a $C^k$ map between manifolds and
  that for some $p \in M$, the derivative ${\bf D}_pf$ is surjective with $\dim
  M = m > n = \dim M'$.
\end{quote}

\begin{itemize}
\item[\textbf{(a)}] Show similarly to Ex VII.2.7 that $p$ and $f(p)$ have charts
  around them giving $f$ the local form
  \begin{align}
    (x^1,\ldots, x^{m-n}, x^{m-n+1},\ldots,x^m) \mapsto (x^{m-n+1},\ldots,x^m).
  \end{align}

\item[\emph{Comment}] As with Ex VII.2.7, this is easier to see with a concrete
  example.  Taking the same spaces used in VII.2.7b-d, let $M =
  S^1\times\mathbb{R}$ (unit cylinder) and $M' = S^1$ (unit circle).  Let $f
  \colon (p^1,p^2,p^3) \mapsto (q^1,q^2)$, and per the hint, $F \colon
  (p^1,p^2,p^3) \mapsto (q^1,q^2,q^3 )$.

  A less obvious example is a mapping from $M = S^2\backslash\{x^3 = \pm 1\}$ (unit
  sphere with poles taken out) to $M' = S^1$ (unit circle) where $f$ takes each
  slice in the $(x^1,x^2)$-plane to the corresponding azimuthal position on the
  unit circle.  The corresponding $F$ mapping would take $M$ into a unit-radius
  cylinder spanning the range $(-1,1)$ in the third coordinate.

  In both cases, the local form of $f$ then drops the extra dimension.

\item[\textbf{(b)}] Deduce that if for some $q \in M'$, every $x \in
  f^\leftarrow(q)$ has ${\bf D}_pf$ surjective, then a chart giving coordinates
  $(x^1,\ldots,x^{m-n})$ of $f^\leftarrow(q)$ may be constructed around each $p
  \in f^\leftarrow(q)$.  Prove that these make $f^\leftarrow(q)$ into a $C^k$
  manifold by satisfying Mi -- Miii.

\item[\emph{Comment}] Note that $f^\leftarrow(q)$ is a subset of $M$ -- the set of
  points in $M$ that $f$ maps to $q \in M'$.  In the examples proposed in part
  (a), we are left with $\mathbb{R}$ or $(-1,1) \subset \mathbb{R}$, which are
  clearly 1-dimensional charts as desired.

  In the general case, we need to make use of the fact that the derivative is
  surjective so show that the slices of $M$ formed by $f^\leftarrow(q)$ satisfy
  Mi -- Miii.  If ${\bf D}_pf$ were \emph{not} surjective, what would it look
  like, and how would that change the result?  I suspect that a deep
  understanding of this question requires understanding how it breaks if the key
  assumption is taken away, but I do not have that understanding right now.

 
\item[\textbf{(c)}] Deduce in particular that if $f\colon \mathbb{R}^n
  \rightarrow \mathbb{R}$ is $C^\infty$ and has ${\bf D}_pf \neq 0$ for every
  $p\in f^\leftarrow(1)$, then $f^\leftarrow(1)$ has the structure of a smooth
  $(n-1)$-manifold.  Construct such functions $f$ to deduce with less work than
  in Ex VII.2.1 that the sets there given in a,b,c are manifolds.
\item[\emph{Comment}]  If we believe part b, the deduction is elementary.
  \begin{enumerate}
  \item[a.] Let $M = \mathbb{R}^3$ and $f\colon (p^1,p^2,p^3) \mapsto
    \sqrt{(p^1)^2 + (p^2)^2 + (p^3)^2}$.
  \item[b.] Let $M = \mathbb{R}^3$ and $f\colon (p^1,p^2,p^3) \mapsto
    \sqrt{-(p^1)^2 + (p^2)^2 + (p^3)^2}$.
  \item[c.] (unit rod in plane) Let $M = \mathbb{R}^4$ and $f\colon (p^1,p^2,p^3,p^4) \mapsto
    \sqrt{(p^3)^2 + (p^4)^2}$.
  \item[c.] (unit rod in $\mathbb{R}^3$) Let $M = \mathbb{R}^6$ and $f\colon
    (p^1,p^2,p^3,p^4,p^5,p^6) \mapsto \sqrt{(p^4)^2 + (p^5)^2 + (p^6)^2}$.
  \end{enumerate}

\end{itemize}
\end{document}



%% commonly used blocks
\begin{quote}
\end{quote}

\begin{proof}
\end{proof}

\begin{align}
\end{align}

\begin{equation}
\end{equation}

\begin{remark}
\end{remark}

