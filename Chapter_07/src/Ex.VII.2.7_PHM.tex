\documentclass[11pt]{article}
% default ORG mode 
\usepackage[utf8]{inputenc}  \usepackage[T1]{fontenc} \usepackage{fixltx2e}
\usepackage{graphicx}        \usepackage{longtable}   \usepackage{float}
\usepackage{wrapfig}         \usepackage{soul}        \usepackage{textcomp}
\usepackage{marvosym}        \usepackage{wasysym}     \usepackage{latexsym}
\usepackage{amsmath}         \usepackage{amssymb}     \usepackage{color}
\usepackage{rotating}        \usepackage{hyperref}    \usepackage{xcolor} 
\usepackage{palatino}
% my customizations
\usepackage{enumitem}
\usepackage{amsthm}
\tolerance=1000
%\providecommand{\alert}[1]{\textbf{#1}}
\usepackage[top=1in, bottom=1in, left=1in, right=1in]{geometry}

\title{Commentary on Dodson \& Poston Exercise VII.2.7}
\author{Peter Mao, $\ldots$}
\date{\today}

\begin{document}

\maketitle
\pagestyle{empty}

\begin{abstract}
  Not a complete solution to this problem, just comments and sketches of concrete
  examples to illustrate the problem.  Feel free to add/comment/disparage.
\end{abstract}

\section{Commentary on Ex. VII.2.7}
D\&P use $x$ both as the point in the manifold $M$ and as the coordinate labels
in the affine spaces that the charts in $M$ and $M'$ map to.  This leads to some
confusion in reading the problem, so in this document, I take $p \in M$, $N$ is
a neighborhood of $p$ and leave the $x^i \in X, X'$ as coordinate in the affine
spaces.

\begin{quote}
  Suppose that $f \colon M \rightarrow M'$ is a $C^k$ map between manifolds and
  that for some $p \in M$, the derivative ${\bf D}_pf$ is injective.  Let $\dim M
  = m \le n = \dim M'$.
\end{quote}

\begin{itemize}
\item[\textbf{(a)}] Deduce from Corollary VII.1.05 that $p$ has a
  neighborhood $N$ such that $f|_N$ is injective.

\item[\emph{Comment}] Corollary VII.1.05 establishes that $f|_N$ is injective
  when the domain and range of $f$ are affine spaces.  We need show that this
  result extends to manifolds as well.  The homeomorphic chart-maps connect the
  manifolds to affine spaces, but we need to show that ${\bf
    D}_{\psi(p)} (\phi\circ f\circ \psi^{\leftarrow})$ is injective.

  By the chain rule,
  \begin{align}
    {\bf D}_{\psi(p)}(\phi\circ f\circ\psi^{\leftarrow}) = {\bf D}_{f(p)}\phi \circ
    {\bf D}_pf \circ {\bf D}_{\psi(p)}\psi^{\leftarrow},
  \end{align}
  so we need to show that ${\bf D}_{\psi(p)}\psi^{\leftarrow}$ and {\bf
    D}_{f(p)}\phi$ are both injective.

  Recall that since we are talking about linear transformations here,
  ``injective'' means that the determinants of these derivatives are non-zero.


\item[\textbf{(b)}] Construct a chart $\phi \colon U \rightarrow \mathbb{R}^n$
  around $f(x)$ such that
  \begin{align}
    \phi(f(N)) = \phi(U) \cap \{(x^1, \ldots, x^n) | x^{m+1} = x^{m=2} = \cdots
      \ x^n = 0\}.
  \end{align}
\item[\emph{Comment}] Note that $(U,\phi)$ is a chart on $M'$ and $f(N) \subset
  U$.  From part (a), we know that $f|_N$ is injective.  It is useful at this
  point to consider what this looks like, concretely.

  \emph{Example}: Let $M = S^1$, the unit circle, and $M' = S^1 \times
  \mathbb{R}$, a unit cylinder.  Let $f \colon (p^1,p^2) \mapsto
  (p^1,p^2,q^3)$, so the unit circle is mapped by $f$ to a particular cut of the
  cylinder.  In $M'$, let $U = \{(x^1,x^2,x^3) | x^2 > 0\}$ and $\phi(U) =
  \{(x^1,x^2=0,x^3)\}$.  To satisfy the condition, define the chart map as
  \begin{align}
    \phi \colon (x^1, x^2, x^3) \mapsto (x^1, 0, x^3 - q^3)
  \end{align}

%%   \emph{Example 1}: Let $M = S^2$, the manifold in Ex VII.2.1a (sphere in
%%   $\mathbb{R}^3$), and $M' = S^2 \times \mathbb{R}^3$, the manifold in Ex
%%   VII.2.1c (unit rod in $\mathbb{R}^3$).  $f$ takes $S^2 \in M$ and sends it to
%%   $S^2 \in M'$ at \emph{a particular location} in $\mathbb{R}^3 \in M'$.



\item[\textbf{(c)}] Show that if $B\colon \mathbb{R}^n \rightarrow \mathbb{R}^m
  \colon (x^1,\ldots,x^m,\ldots,x^n) \mapsto (x^1,\ldots,x^m)$, then $B\circ
  \phi\circ f$ is a chart map admissible on M.
\item[\emph{Comment}] Since $B$ is simply a projection out of the extra
  dimensions carried by $M'$,
  \begin{align}
    \psi = B\circ  \phi\circ f \colon M \rightarrow \mathbb{R}^m
  \end{align}
  is $C^k$ and is a chart map admissible on an open set of $M$ containing $N$.

\item[\textbf{(d)}] Deduce that $x$ and $f(x)$ have $C^k$ charts around them
  which give $f$ the local coordinate form
  \begin{align}
    (x^1,\ldots,x^m) \mapsto (y^1,\ldots,y^m,0,\ldots,0)
  \end{align}
\item[\emph{Comment}] The chart maps $\psi$ and $\phi$ from parts a-c give $f$
  the desired local coordinate form $\phi \circ f \circ \psi^\leftarrow$.

\end{itemize}
\end{document}



%% commonly used blocks
\begin{quote}
\end{quote}

\begin{proof}
\end{proof}

\begin{align}
\end{align}

\begin{equation}
\end{equation}

\begin{remark}
\end{remark}

