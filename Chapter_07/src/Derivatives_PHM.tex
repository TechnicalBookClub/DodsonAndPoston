\documentclass{article}
\usepackage{palatino}
%\usepackage{rotate}
%\usepackage{lscape} %\begin{landscape} \end{landscape}
\usepackage[top=1.5in, bottom=1.5in, left=1in, right=1in]{geometry}


\begin{document}
\title{Derivatives in differential geometry}
\author{Peter H. Mao}
\date{\today}
\maketitle
\begin{abstract}
  This is my attempt to bridge the nomenclature and notations of Dodson and
  Poston, Tu (\emph{Introduction to Manifolds}), Lee (\emph{Introduction to
    Riemannian Manifolds}), Needham and O'neill.

\end{abstract}

\section{Introduction}

\section{Derivative or differential of a map}

Dodson \& Poston introduce the \emph{derivative} in VII.1.01 (pg 151) with the
notation $D_x f$ as the derivative of map $f$ at the point $x$ in the domain of
$f$.  In VII.1.03 (pg 154), they introduce partial derivatives and note that the
Jacobian matrix, consisting of partial derivatives at $x$, is a representation
of $D_x f$.  D\&P distinguish the derivative from the \emph{differential of a
  map} (VII.3.01, pg 171 and VII.3.02, pg 174), denoted
$Df\colon TM \rightarrow TN$ (where $f\colon M \rightarrow N$).  Whereas the
derivative is a map between vector spaces, the differential is a map between the
collections of tangent spaces of domain and range manifolds (which are
themselves also manifolds!).  The derivative is then the (domain) restriction of
the differential to the tangent space of a particular point. (``We are just
taking all the derivatives at once to make one big map.'')  In the special case
where $f$ is a real-valued function, D\&P use the notation $df$ in place of $Df$
for the differential, which they claim \emph{is} standard usage (pg 174).

Tu introduces the \emph{differential of a map} in Section 8.2 (pg 87), with $F$
as the map between manifolds and $p$ denoting a point in the domain manifold,
with the notation $F_*(p)$ or $F_{*,p}$.  The Jacobian matrix is mentioned in
the same section on pg 88, as the local-coordinate representation of the
differential.  Tangent bundles are covered in Section 12 (pg 129), but unlike
D\&P, Tu makes no use of differentiation in this section.

Lee skips over this topic and goes straight into the Koszul connection (pg 85).

Needham first introduces derivatives on pg 42 with no formal definition.  He
indicates the derivative of a map $f$ with $f'$.

\section{Partial derivative}

Dodson \& Poston introduce the \emph{partial derivative} in VII.1.03 (p 153) as
the components of the derivative once we have chosen charts.\footnote{The term
  ``partial derivative'' is first used on p 150.}  The notation used in D\&P is
the uncontroversial $\partial/\partial x^i$  or $\partial_i$.  (Note to self:
sort out the whole vector/covector business in this notation.) Tu introduces
this topic in section 6.6 (p 67) using the same notation.  Lee also appears to
use the same notation, but does not specifically define the partial derivative.

The Jacobian matrix is composed of partial derivatives of components of the
function for which we are taking the derivative (D\&P p 154).

Needham only mentions the Jacobian (p 375) in the context of using the area
2-form instead of the Jacobian.

\section{Exterior derivative}

The exterior derivative is not covered by Dodson \& Poston because they don't do
forms.  (not true:  Gary has something from Ch V, Barney claims that VII.1.03
introduces the d)

Needham: (p 355) $\bf{d}$, where $\bf{d}f(\bf{v}) = \bf{\nabla_v}f$.
(p 394) $\bf{d}$ raises the degree of a form by 1, e.g. it takes 1 forms into 2
forms, etc.
(p 406) [37.2] exterior derivative of a form is a integeral ?!?
(p 439) Cartan's first structural equation
(p 457) generalized exterior derivative

Tu:


O'neill:


Lee: Covered in Appendex B (p 401).


Bamberg \& Sternberg: Chapters 7,8.



\section{Lie, covariant, or intrinsic derivative}

Identificationos based on geodesic equation
D\&P p 246
Needham p 244, p 284 ($D_v$ (intrinsic) vs $\nabla_v$ (directional)), p 286
(intrinsic derivative, bold nabla).  Note that $D_v$ and $\bf{\nabla}_v$ are the
same object, defined by equations 23.2 and 29.2 respectively.

(Look up definition of unbolded $\nabla$)

\section{Directional derivative}

Same as exterior derivative? [RP]

[JV] Only in the special case of the exterior derivative of a scalar field.

[GL] df(\partial_x) = \frac{\partial f}{\partial x}, hence Needham 32.10.


\section{velocity along curve}

\end{document}
